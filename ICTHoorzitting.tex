\documentclass[10pt]{article}   	% use "amsart" instead of "article" for AMSLaTeX format
\usepackage{geometry}                		% See geometry.pdf to learn the layout options. There are lots.
\usepackage[dutch]{babel}
\geometry{a4paper}                   		% ... or a4paper or a5paper or ... 
%\geometry{landscape}                		% Activate for for rotated page geometry
%\usepackage[parfill]{parskip}    		% Activate to begin paragraphs with an empty line rather than an indent
\usepackage{graphicx}				% Use pdf, png, jpg, or epsß with pdflatex; use eps in DVI mode
\usepackage{fontspec}
\usepackage{multicol}
%\setmainfont{gill sans}
					
\linespread{1.1}			% TeX will automatically convert eps --> pdf in pdflatex		
\usepackage{amssymb}
\usepackage{eurosym}
\usepackage{colortbl}
\title{Negatieve spiralen in ICT ontwikkeling bij de overheid}
\author{Stephan J.C. Eggermont, Sensus}
\begin{document}
%\pagestyle{empty}
\newcommand{\logo}{\fontsize{31}{31}\selectfont}
\newcommand{\logosmall}{\fontsize{7.2}{7.2}\selectfont}

\setlength{\parindent}{0pt}
\maketitle
\begin{quote}
\em
De ICT hoorzittingen die afgelopen jaar gehouden zijn door een commissie van de tweede kamer, voorgezeten door
Dhr. Elias, geven een interessant, en voor sommigen wellicht schokkend, beeld van hoe de Nederlandse 
overheid omgaat met het verwerven van ICT voorzieningen. De geïnterviewden gaven ieder hun eigen
beeld en benoemden een deel van de problemen. Er werden heel veel verschillende problemen (en 
mogelijke oplossingen) genoemd. Als deze in één plaatje (een zogenaamd diagram 
of effects\cite{Weinberg1991}) worden neergelegd, en gecombineerd met
oorzaak-en-gevolg-pijlen, worden meerdere negatieve spiralen zichtbaar. Er is geen probleem dat in alle
spiralen voorkomt. Daardoor is duidelijk dat het oplossen van de problemen niet met één enkele maatregel 
kan gebeuren. Bovendien is causaliteit niet meer eenduidig (vooraf) te bepalen:
er zitten allerlei vertragingen in het systeem die het instabiel
maken. Een kleine ingreep op één punt kan ergens anders in het systeem
tot grote veranderingen leiden.\cite{SnowdenBoone2007} 
\end{quote} 

\begin{multicols}{2}
\includegraphics[width=\columnwidth]{aanbesteding.pdf}
{\em \center  Figuur 1. Een (deel van de) aanbestedingsspiraal}

\section{Spiralen}
Negatieve spiralen hebben als kenmerk dat er een zelfversterkend
effect van uitgaat. Ze maken het systeem instabiel, en er zit
vertraging en hysterese in. Het is ook niet
zinvol één element als oorzaak aan te wijzen. De keuze voor een beginpunt, en de naamgeving van een 
spiraal is hier dan ook voornamelijk bepaald door het gemak waarmee de spiraal kan worden verklaard.
Het is zonder twijfel mogelijk om deze spiralen vanuit een ander
startpunt te beschouwen. 
Er zijn veel spiralen te vinden in de interviews. Hier worden er slechts enkele uitgewerkt:
\begin{itemize}
\item Focus op efficiency
\item Aanbesteding
\item Uitbesteding
\end{itemize}

\section{Focus op efficiency}

De focus op efficient werken leidt tot vraagbundeling en centralisatie. Immers, door vraagbundeling hoeft
maar een systeem gebouwd te worden i.p.v. meer dan een, en door centralisatie neemt de complexiteit van
projecten toe. Door de toegenomen complexiteit, neemt de doorlooptijd toe. Doordat de doorlooptijd
toeneemt, neemt het aantal veranderingen toe. Om een project goedgekeurd te kunnen krijgen door de tweede
kamer is het belangrijk om een zo laag mogelijk budget nodig te hebben. Door de vraagbundeling wordt het
moeilijker om te zien of een project compleet is. Als een systeem moet aansluiten op een ander systeem
kunnen de benodigde wijzigingen over het algemeen in beide systemen worden doorgevoerd. Als de beide
systemen niet onder één budgethouder vallen, kan gemakkelijk de aanname gedaan worden door allebei dat
de ander zijn systeem wel zal aanpassen. Bij het doen van een projectaanvraag is de verleiding heel groot om
benodigde wijzigingen in aansluitende systemen niet op te nemen in de begroting. Door vraagbundeling 
kan ook de situatie ontstaan dat er geen expert is die het geheel overziet, en de experts op de deelterreinen
simpelweg niet weten waar nog meer wijzigingen moeten worden uitgevoerd.

Bij het implementeren van het project wordt dan duidelijk dat er meer werk moet plaatsvinden dan oorspronkelijk
begroot. Om niet de schuld te krijgen van het mislukken van een project worden tijdelijke medewerkers aangetrokken
die de verantwoordelijkheid voor het project krijgen tijdens fases die als risicovol gezien worden. Doordat het
project op meer systemen moet aansluiten dan oorspronkelijk voorzien, worden meer partijen betrokken 
bij het project. De keuze voor welk systeem moet worden aangepast wordt niet genomen op technische criteria,
maar op basis van waar nog een potje met geld beschikbaar is, zodat de minister zo min mogelijk naar de
kamer moet voor extra geld. Verdere vraagbundeling voor het project maakt het ook makkelijker om meer geld 
ter beschikking te krijgen, en maakt het mogelijk om risico's verder vooruit te schuiven, bij voorkeur tot er 
een nieuwe minister is.


\section{Aanbesteding}

Grotere ICT projecten moeten worden aanbesteed. Het uitvoeren van een aanbesteding is veel werk.
Om de overhead van een aanbesteding niet te veel te laten drukken op de totale projectkosten is
het handig om projecten groot te maken. Inschrijvende partijen besteden gemiddeld enkele weken 
met een klein team aan het maken van een projectvoorstel, en doen dat onbetaald. Er zijn enkele
grote partijen die meedoen aan aanbestedingen en de kans om een aanbesteding te winnen is
enkele tientallen procenten. Bij een aanbesteding moet vooraf worden vastgelegd wat de aanbestedende
partij nodig heeft. Het maken van inschattingen voor softwareontwikkeling is moeilijk. Bij de meeste
overheidsaanbestedingen wordt dit gemaskeerd doordat slechts een deel van de aanbesteding 
gaat over softwareontwikkeling. Softwareontwikkeling inschatten is onder andere moeilijk
omdat het gaat om innovatie. Er is geen bruikbare manier bekend om grotere projecten in te schatten,
en hoe groter hoe moeilijker. Als een project groter wordt zijn er meer partijen bij betrokken en
wordt besluitvorming hoe langer hoe meer politiek en niet technisch. Ook wordt de kans steeds groter 
dat er een deel in zit dat niet bouwbaar is.  Door het aanbestedingsproces moet er vroegtijdig 
geschat worden. De kwaliteit van schattingen wordt beter naarmate er later geschat wordt. 
De inschrijvende partij zal risico's zoveel mogelijk bij de aanbestedende partij willen leggen,
en kan alleen inschatten waar hij zelf verstand van heeft. Om zo goedkoop mogelijk aan te bieden, 
zal hij alles wat niet eenduidig is vastgelegd, in zijn voordeel uitleggen. Daardoor moet het
aanbestedingsdocument veel uitgebreider en preciezer zijn dan technisch verantwoord. 
Er moeten al keuzes gemaakt worden om een onderdeel al dan niet binnen de aanbesteding te
laten vallen, voordat bekend kan zijn of dat de handigste en goedkoopste oplossing is. 


\section{Uitbesteding}

Door het uitbesteden van ICT ontwikkelwerk naar private partijen ontstaat een cyclus waarbij
het kennisniveau van overheidsmedewerkers afneemt. Er ontstaat een scheiding tussen 
uitvoerend ontwikkelwerk en het beschrijven van programma's van eisen/aanbestedingen en architectuur.  
Uitvoerend ontwikkelwerk is laag ingeschaald en goede technici kunnen veel meer verdienen bij 
private partijen. Dat zorgt voor een steeds groter kwaliteitsverschil tussen technici bij 
overheid en bedrijfsleven. Door het ontbreken van technische expertise bij de overheid neemt de
technische kwaliteit van aanbestedingen af. 

Als het technisch beoordelen van aanbiedingen daarmee
ook in kwaliteit afneemt, ontstaat een steeds grotere neiging aanbiedingen alleen op prijs te 
beoordelen, en daarmee de laagste inschrijver te selecteren. Private partijen zullen als gevolg daarvan 
de aanbestedingen steeds meer minimaal gaan uitleggen, om een zo laag mogelijke prijs te kunnen
aanbieden. Als ze technische gebreken in een aanbesteding ontdekken ontstaat een dilemma:
als ze die ignoreren weten ze dat meerwerk nodig gaat zijn, maar dat er ook een conflict over 
ontstaat. Daardoor kunnen alleen private partijen aanbieden die een voldoende grote juridische
afdeling hebben om deze conflicten te kunnen winnen. Als ze de technische gebreken openbaar maken,
maken ze duidelijk dat de aanbestedende partij fouten gemaakt heeft. Dat is niet bevorderlijk voor
de kans om de aanbesteding gegund te krijgen. Door de manier waarop aanbestedingen vormgegeven
zijn, kunnen ze de gebreken alleen met alle concurrenten delen. In de praktijk wordt de vragenronde
in het aanbestedingsproces dan ook alleen kritisch gebruikt om aanbestedingen te stoppen.  

Doordat steeds conflicten ontstaan, wordt het beschrijvende ICT werk ook steeds minder interessant. 
Het wordt noodzakelijker om niet verantwoordelijk gehouden te kunnen worden voor de tijd- en
budgetoverschrijdingen die het gevolg zijn van de conflicten. Omdat technische kennis steeds minder
aanwezig is, wordt het steeds moeilijker om te toetsen of wat beschreven is in de aanbesteding wel
veilig te vragen is. Daarmee wordt ook op beschrijvend niveau een baan bij een private partij aantrekkelijker.

\section{Oplossingen}
Om tot een oplossing voor deze problemen te komen moeten deze spiralen gelijktijdig aangepakt worden.
Op basis van bovenstaande lijst zijn de volgende maatregelen te nemen:
\begin{itemize}
\item De complexiteit moet systematisch worden teruggedrongen door het verkleinen en opsplitsen van 
projecten. De verantwoordelijkheid, met bijbehorende budgetten, moet worden neergelegd zo dicht
mogelijk bij het niveau waar de software gebruikt moet gaan worden. 
\item Werk dat uniek voor de overheid is moet niet worden uitbesteed. Er is een salarisstructuur noodzakelijk
die het aantrekkelijk maakt voor academisch opgeleide programmeurs om voor de overheid te werken.
\item In plaats van vraagbundeling in enorme projecten moet een infrastructuur gecreëerd worden die
schaalverkleining mogelijk maakt, zodat effectief en efficient maatwerksoftware gebouwd kan worden.
\item Projecten moeten gestuurd worden op waarde en risico. Fasering moet zo zijn dat risicovolle delen 
vroeg in het traject plaatsvinden, en dat eindgebruikers zo snel mogelijk de software kunnen gebruiken.
\end{itemize}
\begin{thebibliography}{9}

\bibitem{Weinberg1991}
  Gerald M. Weinberg (1991),
  \emph{Quality Software Management, Volume 1, Systems Thinking}.
  Dorset House Publishing.

\bibitem{SnowdenBoone2007}
  Snowden, David \& Boone, Mary (2007),
  \emph{A Leader's Framework for Decision Making}.
  Harvard Business Review, November 2007: 69-76.

\end{thebibliography}


\end{multicols}



\end{document}  